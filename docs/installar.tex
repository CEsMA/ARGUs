\documentclass[spanish,10pt]{article}
\usepackage[hmargin=1cm,vmargin=1cm]{geometry}
\usepackage{verbatim,babel}
\usepackage[utf8]{inputenc}

\title{Recetario para la instalación de Argus}
\author{Fabio Castro y Andrés Sajo}
% \date{}

\begin{document}
\maketitle
\section{Ingredientes}
\subsection{aplicaciones}
Los paquetes necesarios para la instalación.
\begin{itemize}
\item[] Ubuntu 14.04 ó Debian 7.0
\item[] Ruby 1.8
\begin{itemize}
\verb+# apt-add-repository ppa:brightbox/ruby-ng+ (Solo Ubuntu)
\verb+# apt-get update+
\verb+# apt-get install ruby1.8+ (Debian y Ubuntu)
\end{itemize}
\item[] Postgresql 8.4 \footnote{funciona en esta versión.} 
\begin{itemize}
\item[] \verb+# touch /etc/apt/sources.list.d/pgdg.list+
\item[] \verb+# echo "deb http://apt.postgresql.org/pub/repos/apt/ precise-pgdg main" | sudo tee /etc/apt/sources.list.d/pgdg.list+
\item[] \verb+# wget --quiet -O - https://www.postgresql.org/media/keys/ACCC4CF8.asc | sudo apt-key add -+
\item[] \verb+# apt-get install postgresql-8.4+
\end{itemize}
\item[] \verb+# apt-get install rubygems+\footnote[2]
\item[] \verb+# apt-get install libecpg-dev+\textsuperscript{2}
\item[] \verb+# apt-get install libmagickwand-dev+\textsuperscript{2}
\item[] psql\footnote{instalado automáticamente al instalar postgresql-8.4.}
\end{itemize}
\subsection{gems}
Las gemas se encuentran en el directorio \texttt{install/gems}. Nota: hay gemas que se requien en versiones múltiples.
\verbatiminput{gemas_completa.txt}
\section{Instalación}
Instalación de las gemas: Una por una y tomando en cuenta la lista de dependecias.
\begin{verbatim}
 gem install --local postgres-0.7.9.2008.01.28.gem
 gem install --local NOMBRE_DE_GEMA-VERSIONGEMA.gem
\end{verbatim}

\section{Configuración}


\subsection{Postgres}
Construir tabla, cambiar \emph{encoding} y vaciar el \emph{dump} de la base de datos. La partición donde se encuentra el archivo 
\verb+dump_hidrox.sql+ debe tener almenos 10GB libre.
\begin{verbatim}
sudo su postgres; 
psql;
postgres=# CREATE DATABASE hidro_development;
postgres=# \l
postgres=# update pg_database set encoding=8 where datname='hidro_development';
postgres=# \l
postgres=# CREATE USER tepedino WITH PASSWORD 'tesis';
postgres=# GRANT ALL PRIVILEGES ON DATABASE "hidro_development" to tepedino;
postgres=# \c hidro_development
postgres=# SET CLIENT_ENCODING TO 'latin1';
postgres=# \i ./database/dump_hidrox.sql
postgres=# \q
\end{verbatim}
\subsection{Archivos de Configuración de Argus}
\item[] cp ./install/mongrel.sh /etc/init.d
\item[] update-rc.d mongrel.sh defaults
\begin{verbatim}
 tar zxvf ./repo/Argus.tgz -C /opt/argus
\end{verbatim}
o
\begin{verbatim}

 git clone git@github.com:fabiocasmar/ARGUs.git
\end{verbatim}
\subsection{Mongrel Web server}
Iniciar el servicio de Argus: 
\begin{verbatim}
cd /home/argus/
service mongrel.sh start
\end{verbatim}
Luego de este paso se apunta el navegador al ip de la maquina, \emph{e.g.} \texttt{http://159.90.14.201/}.
\end{document}
